\section{Elastic Joint Robots: Dynamic Modeling}\label{sec:Dynamic Modeling}
To accurately model the elastic joints some standard assumptions are needed, we'll base our dissertation on \textbf{A1, A2, A3} and the simplification \textbf{A4} (true engaging large reduction ratios), cleverly 
described in \cite{13sic} \textit{section 13.1.1} and summarized in Figure~\ref{fig:assumpt}, so as to ensure long life of electrical drives, linear elasticity, decoupling and independence properties.
% want more details?
\subsection{Reduced Model}
In contrast with the rigid robot, here there is a displacement between motor and link frames, this requires to take into account $2N$ moving rigid bodies. Hence, a very convenient choice is the use of $2N$ generalized coordinates, $N$ for the link $\qv$ and $N$ for the motor $\thetav$ positions. Paying attention to the fact that the rotors contribute to the kinetic energy and the joint deflections add an elastic component to the total potential energy, the usual Euler-Lagrange method can be applied 
yielding:
\begin{equation} \label{eq:reduced}
    \begin{pmatrix} 
    \Mm\!(\qv)\ddqv+\cv(\qv,\dqv)+\gv(\qv)+\Km(\qv-\thetav)\\ \Bm\ddthetav+\Km(\thetav-\qv) 
    \end{pmatrix} 
    =
    \begin{pmatrix} \zerov \\ \uv \end{pmatrix}
\end{equation}
where $\Bm$ is the constant diagonal inertia matrix collecting the rotors inertial components around their spinning axes, $\Mm\!(\qv)$ is the sum of the link inertia matrix $\Mm_{\!L}$ and $\Mm_{\!R}$ which contains the rotor masses and the other rotor components, and $\Km$ $>\zerov$ the diagonal matrix of joint stiffness. Equations~(\ref{eq:reduced}) are also known as reduced model.
\subsection{Complete Model}
Removing the simplification, we have to consider also the inertial couplings between the rotors and the previous links in the robot chain under the square matrix $\Sm(\qv)$, attaining the complete dynamic model:
\begin{equation}\label{eq:completemod}
    \begin{pmatrix}\Mm\!(\qv)~\Sm(\qv)\\ \Sm^T\!(\qv)~\Bm\end{pmatrix}
    \begin{pmatrix}\ddqv\\ \ddthetav\end{pmatrix} +
    \begin{pmatrix}\cv(\qv,\dqv)+\cv_1(\qv,\dqv,\dthetav)\\ \dcv_2(\qv,\dqv)\end{pmatrix} +
    \begin{pmatrix}\gv(\qv)+\Km(\qv-\thetav)\\ \Km(\thetav-\qv)\end{pmatrix} =
    \begin{pmatrix} \zerov \\ \uv \end{pmatrix}
\end{equation}
moreover in presence of energy-dissipating effects, the non-conservative generalized forces appear on the right side of (\ref{eq:completemod}), for example viscous friction $\Fm_{\!\theta}$, $\Fm_{\!q}$ of the two sides of the transmissions and spring damping $\Dm$ of the joints (all matrices diagonal and positive definite) give rise to the term:
\begin{equation}
    \begin{pmatrix} -\Fm_{\!q}\dqv-\Dm(\dqv-\dthetav) \\
    -\Fm_{\!\theta}\dthetav-\Dm(\dthetav-\dqv)
    \end{pmatrix}
\end{equation}
Eventually, it's interesting to note that when $\Km \rightarrow \infty $, then $\thetav \rightarrow \qv $ while the elastic torque $\tauv_{\!j}=\Km(\thetav-\qv) \rightarrow \uv$, making (\ref{eq:completemod}) collapse into the standard fully rigid model.